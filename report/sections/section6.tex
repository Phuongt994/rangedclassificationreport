\chapter{Conclusion}

In conclusion, the program implements the original algorithm, from which it is able to derive classification rules given a set of user-specified thresholds. With minor modifications to range adjustment and adaptation to Java data structure, the program is in shape to produce outcome in the required format.

However, the given output is suggested to be inherently inaccurate: output rules do not vary as expected, and at very low threshold ($0.1$ when tested), potential crashes could occur. It is suspected that there exist structural errors, as data partitions in second iteration onwards might have been passed on incorrectly within the recursive call, resulting in wrong output. Because of inheritence call also being nested in recursive loop, bad inner calls is difficult to be spotted. 

Nevertheless, the program could produce correctly formatted output and with more time allowance, systematic bugs can be tested and amended. It is believed that a revision to parameters passed within the recursive loop could redirect the algorithm to work on the right resources and put the program on track. 