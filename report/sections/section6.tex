\chapter{Conclusion}

In conclusion, the programme implements the original algorithm, from which it is able to derive classification rules given a set of user-specified thresholds. With minor modifications to range adjustment and adaptation to Java data structure, the programme is in shape to produce outcome in the required format.

However, the given output is suggested to be inherently inaccurate: output rules do not vary as expected, and at very low threshold ($0.1$ when tested), potential crashes could occur. It is suspected that there exists structural errors, as data partitions in the generating phase might have been passed on incorrectly, resulting in wrong output. Moreover, crashes occur at lower threshold boundary might concern with a bad behaviour within recursive calls. Because of inheritence calls being nested in recursive loop, bad inner calls might be difficult to be spotted.

Nevertheless, the programme could produce correctly formatted output and with more time allowance, systematic bugs can be tested and amended. It is believed that a revision to parameters passed within the recursive loop could redirect the algorithm to work on the right resources and put the programme on track. 