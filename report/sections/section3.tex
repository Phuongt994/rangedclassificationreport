\chapter{Project Algorithm}

Based on the original algorithm discussed in the previous section, initially this project only aimed to implement the exact algorithm into hard code. However, during the implementation process, several adjustments were agreed to be made. Some adjustments (e.g additional steps in scanning data input, display final rules to screen) are for the neccessary code adaption i.e. pre-processing data for input and output, at beginning and end respectively. Others (e.g max-sum, threshold check, null-class converter) are for possible improvements in algorithmic in the programme. 

Respectively, for the technical-related changes, section 4 will discuss them in depth where the actual code implementation is concerned. For the algorithmic changes, they are to be discussed in the remaining sections below.

\section{Similarities}

As the project’s main goal is to implement the algorithm, the programme structure is entirely similar to the initial algorithm i.e. Phase I (Analyse) and Phase II (Generate) is kept in order. Details are explained in the following sub-sections. \\


\subsection{Project Phase I (Analyse)}

Phase I contains the range analysis method (max-sum) and the threshold checker (for support, density and confidence). In this project, similar steps are implemented: Firstly, the table of data is divided into sub-data by class value (step 1-2). Secondly, within this sub-data, a $[v_{min}, v_{max}]$ range is derived for each of the attribute (step 3). Thirdly, for each of these ranges within an attribute, a max-sum analysis is conducted so that sub-ranges can be deduced from the original range (step 4-7). Finally, each of these sub-ranges is then checked to see whether it passes the specified thresholds of support, confidence and density (step 8-12). The accepted sub-ranges are collected and sent to Phase II (Generate) (step 13).

However, in this project, several steps are done differently in terms of algorithm compared to the initial algorithm. Firstly, the max-sum analysis is adapted to fit in this range-based classification problem. While it was not discussed in details in the initial paper, there is a difference in this max-sum compared to the original max-sum solution. That is, this max-sum method will not seek for one maximal array but multiple ones depending on a set of pre-specified conditions. Secondly, in the threshold checking part, the order of threshold tested are no longer support - density - confidence, but support - confidence for first iteration and density - support - confidence for the second iteration onwards. It is to adapt to the different structural as well as technical requirements of different iteration e.g. density is not relevant in the first iteration where no extended ranges have been made yet. Details are to be discussed in the next section.

\subsection{Project Phase II (Generate)}

Phase II contains the range combination method (a-priori) and the additional adjustment step (null-class converter). In this project, the first method is implemented but the second part is replaced by an elimination step (eliminator): Firstly, the accepted sub-ranges from Phase I are considered and grouped with each other under the condition that one of them has an extendable range (step 1-7). Secondly, according to the original algorithm, an adjustment method is required by changing the class value of partly-covered ranges to Null (step 8). 

However, instead of turning tuple's class value to $Null$ only, this project adopts a stricter policy: all partly-covered ranges are eliminated and not considered for the next iteration. It is due to the idea taken from the paper that those tuples are deemed to badly influence the rule accuracy in the long run. While they might as well get eliminated in the next max-sum iteration, the algorithm in this project does not at all consider those tuples to save on processing time and enhance the rule accuracy in exchange for less support. Hence the null-class converting process (step 8) is replaced with an elimination process. [SUPPORTINGPAPER?] Details will be discussed in the next section.

\section{Differences}

As mentioned in the previous section, there are two main algorithmic changes in this project compared to the initial algorithm. They are:
[FIG?]

\subsection{Project Phase I (Analyse)} 

\begin{itemize}
	\item Max-sum method $\rightarrow$ maxsum() in detail \\
	\textit{Original version}: Max-sum on multiple ranges - not elaborated on paper \\
	\textit{Project version}: Max-sum on multiple ranges in details - elaborated \\

	\item Threshold checking method $\rightarrow$ separated check() \\
	\textit{Original version}: Threshold check consists of three orderly checks: support-density and confidence \\
	\textit{Project version}: Threshold check consists of two orderly checks for first iteration and three orderly checks for second iteration onwards. The order of each threshold check is also altered (for first iteration: support-confidence; for second iteration: density-support-confidence) \\
\end{itemize}

\subsection{Project Phase II (Generate)} 

\begin{itemize}
	\item Null-class converting method $\rightarrow$ adjust() \\
	\textit{Original version}: a conversion where partly covered tuples have their classes change to Null \\
	\textit{Project version}: a strict adjustment where partly covered tuples are complelely removed and not considered during the next iteration \\
\end{itemize}


\section{Final Construction}

Algorithm here but of course I am cursed to have files deleted before saving.



