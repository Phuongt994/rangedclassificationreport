\chapter{Implementation}

\section{Structure}

\subsection{Overview}

Based on the project algorithm design, its Java implementation unit complies with the specified structure and thus maintain the two core phases ($Analyse$ and $Generate$). However, hard code requires extra handling of data input and output. Hence additional two phases are added: $Scan$ and $Monitor$, which read and display input respectively. Moreover, the differences in data type of different iterations require extra treatment would be too clunky to append to the $Analyse$ phase. Thus, another phase is in line: $Sub-Analyse$, which inherits $Analyse$ range-analysing functions while having additional threshold check (for density) and handle elimination round for partly-covered tuples.  

The implementation process can now be divided into five Phases instead of two from the original algorithm. The reasoning is assessed in details below:

\begin{table}[h]
\caption{5-Phase Implementation and Reasoning}
\label{table:table4_6}
\centering
\begin{tabular}{lc}
	\toprule
	\textbf{Reason} & \textbf{Outcome} \\
	\midrule
	To read and pre-process data input before
analysis & Phase I Scan \\
	To comply with the original structure	& Phase II Analyse \\
	To comply with the original structure	& Phase II Generate \\
	To adapt to different iteration needs	during
data analysing process & Phase IV Sub-Analyse \\
	To monitor and record the result & Phase V Monitor \\
	\bottomrule
\end{tabular} 
\end{table}


From Table 4.1, a detailed structure for the program can be presented in an Activity Diagram below:
\begin{figure}[h]
    \centering
    \includegraphics[width=6.5in, height=6.8in]{figures/activity_overview}
    \caption[Class Overview: Activity Diagram]{Class Overview: Activity Diagram}
    \label{fig:figure4_1}
\end{figure}


 To demonstrate the structure graphically, a Class Diagram and a Sequence Diagram is shown for each phase, that is represented by each class. Class Diagram is chosen as a common representation for an overview within and between classes \cite{seclass}. A Sequence Diagram is also chosen against Activity Diagram since it shows the recursive calls in different iterations in depth.

\subsection{Phase I Scan}

\begin{description}

\item[Aim: ] There are three aims within this phase or class:
\begin{itemize}
	\item{To read data from \texttt{.csv} file} 
	\item{To sort data into sub-data pools using class values found when reading data in} 
	\item{To pass these data into the next phase or class ($Analyser$} 
\end{itemize}

\item[Input: ] This class input is the data itself:
\begin{itemize}
	\item{Pre-processed .csv file (no null values within attributes and the order of element in an instance is {attribute $1$, attribute $2$,..., attribute $N$, class $C$}}
\end{itemize} 

\textit{Reason: } Initial plan included an extra processing unit for filtering csv file when value is Null. However, time constraint did not allow the construction of this unit. Hence the program only consider pre-processed csv data at the time the report is written.

\item[Output: ] This class output are data taken from input source:
\begin{itemize}
	\item{\texttt{AllTuples}: A list of all instances of the data - a list of tuples that are presented as lists.}
	\item{\texttt{AllClassMap}: A partitioned map that contains class values as keys that are connected to their relevant sub-data (list of instances) as values.}
\end{itemize}
  
\textit{Reason: } Data taken from input sources need separate containers for different purpose: \texttt{AllTuples} is required to provide access to all original data. On a different note, \texttt{AllClassMap} is required to get the class-based data partition.

\item[Data type: ] \texttt{LinkedList} (for tuples), \texttt{LinkedHashMap} (for class-based data partitioned map)

\textit{Reason: } A \texttt{List} - without specified inner data type - is used for tuples since it needs multiple types of data (e.g. float number, string class value, integer ordered number). In this case, \texttt{LinkedList} is used instead of \texttt{ArrayList} for two reasons: Firsrly, its ability to add and remove first and last elements more quickly than the alternative. It is necessary to retrieve class value in each tuple at faster rate as class value is the last element in the tuple. Secondly, \texttt{LinkedList} or any type of \texttt{Linked}-collection type means its  A set of tuples thus casts as \texttt{LinkedList<LinkedList>}.

Similarly, a \texttt{LinkedHashMap} is used for faster retrieval rate an. A \texttt{HashMap} is used for storing partitioned sub-data from class values since each partition of data needs to be recorded with its unique class value. It has String class as key and \texttt{LinkedList<LinkedList>} List of all tuples as values. 

\item[Class diagram: ] 

\begin{figure}[h]
    \centering
    \includegraphics[width=4in]{figures/class_scanners}
    \caption[Class Diagram for class Scanners]{Class Diagram for class Scanners}
    \label{fig:figure4_2}
\end{figure}

Its class diagram is in Figure 4.2, showing two features/methods:
\begin{itemize}
	\item{scanData()} \\
	This method initialises two scanners, one to scan line-by-line data from csv file (i.e. tuples) and one to scan in-line elements (i.e. attributes).
	It processes data into 1) allTuple (where all instances are recorded from the first scanner) and 2) Class values, Tuple order numbering and Tuple attributes (which is sorted by the second scanners).
	The Class values become the parameter for the next method, where it helps create the Map that connects Class value to relevant set of Tuples.
	\item{sortData()} \\
	This method groups sets of tuples according to their Class value by checking their last String element. It sorts and partitions allTuple into sub-data pools, which are then passed on as parameter to the new class ($Analyser$).
	
\end{itemize}

\item[Sequence diagram: ] 

Figure 4.3 shows the sequence diagram for $Scanners$, showing the data flow from $Scanner \rightarrow Analyser$:

\begin{figure}[h]
    \centering
    \includegraphics[width=3in, height=3in]{figures/sequence_scanners}
    \caption[Sequence Diagram for class Scanners]{Sequence Diagram for class Scanners}
    \label{fig:figure4_22}
\end{figure}

\end{description}

\subsection{Phase II Analyse}

\begin{description}

\item[Aim: ] Being the core module of the program, there are important aims for this phase or class:
\begin{itemize}
	\item{To find (min, max) range for each attribute from class-based map.} 
	\item{To convert (min, max) range to binary array, which is ready for max-sum analysis.} 
	\item{To perform max-sum analysis on (min, max), aim to obtain at least one sub-optimal range, which is ready for the threshold test.} 
	\item{To perform threshold check. As discussed, this check diverses depending on which iteration. However, support and confidence is always required thus appear in this phase or class.} 
	\item{To modify the new sub-data pools for the next iteration. Since data set would shrink after pruning process from max-sum and (if applicable) strict-adjust method, current data set is frequently changing. Threshold measures, however, are calculated depending on these sets, thus makes this modification step necessary.} 
	\item{To pass the modified sub-data pools as well as new sub-ranges to the next phase or class ($Generator$).} 
\end{itemize}


\item[Input: ] This class input is the original dataset and the class-based data map from passed on from previous class:
\begin{itemize}
	\item{\texttt{allTuple}:  all original instances in dataset.} 
	\item{\texttt{allClassMap}: with key as class tag and value as relevant sub-data partition.}
\end{itemize}

\textit{Reason: } \texttt{allClassMap} is needed for the class-guided search for the rest of the program. This map, however, changes as the sub-data pool changes throughout the cycle. Therefore, \texttt{allTuple} is needed to refer back to the original, pre-modification input when necessary.

\item[Output: ] A list of sub-ranges in float values that passed the support-confidence check (\texttt{attributeRangeList}). It also returns a modified version of allClassMap, which \texttt{AllClassMap} is now divided into two maps: \textit{attributeRange} and \textit{attributeTuple}.
  
\textit{Reason: } As required for the next $Generate$ phase, \texttt{attributeRangeList} stores potential associated ranges for new combination. 
Map, however, diversifies into two: 
\begin{itemize}
	\item One for storing sub-ranges (that are also stored in \texttt{attributeRangeList})for each attributes. That means \texttt{attributeRangeMap} has its key quantity equal to the number of attributes, and each key maps to at least one sub-range found from analysing methods.
	
	\item One for storing tuples that each sub-range in \texttt{attributeRangeMap} covers. That means \texttt{attributeRangeMap} has its key quantity equal to the number of sub-ranges available (that are also recorded in \texttt{attributeRangeList}), and each key maps to a list of tuples that is covered by that range.
	
Initially, only \texttt{attributeRangeMap} was implemented as it was thought that only attribute and its sub-ranges were necessary to be recorded formally. Tuples covered by sub-ranges were thought to be retrievable by tracing back to the original \texttt{AllClassMap}.
	
However, as data was partitioned further the longer the algorithm persisted, there was no fixed map that remained relevant for the entire duration of the program. With the dynamic nature of tuple sets, it was safer to recorded them right when their sub-ranges were being processed. Thus there was a need for the additional \texttt{attributeTupleMap}.
\end{itemize}

\item[Data type: ] For \texttt{attributeRangeMap}, a \texttt{LinkedHashMap} with \texttt{LinkedList<Integer>} as keys and \texttt{LinkedList<LinkedList<Float[]>>} as values is used. This map stores attribute number as key (thus \texttt{Integer} type cast) and ranges as value (thus \texttt{LinkedList<Float>} type cast). It must be noted that despite attribute number being a singular integer (from $1$ to $n$), \texttt{LinkedList} is implemented since attributes' ranges are combined at Generate phase. That means attribute number must be expandable to manage this new combination (e.g attribute $1$ and $2$ has associated ranges of $[1.1,1.5]$ and $[2.2, 2.6]$. Once these two ranges are combined to form larger ranges, attribute numbers are also combined i.e. to ${1,2}$ to manage this combined range in the map) - hence need for a \texttt{List}. Likewise, \texttt{LinkedList<Float>} is sufficient to manage a single range; when there are multiple ranges to be stored, an outer \texttt{LinkedList} must be used.

Similarly, For \texttt{attributeTupleMap}, a \texttt{LinkedHashMap} with \texttt{LinkedList<LinkedList<Float[]>>} as keys and \texttt{LinkedList<LinkedList>} as values. Keys in \texttt{attributeTupleMap} are \texttt{attributeRangeMap} values; the cross-reference makes for a convenient key retrieval between two maps. 

\item[Class diagram: ] 

\begin{figure}[h]
    \centering
    \includegraphics[width=4.5in, height=4in]{figures/class_analyser}
    \caption[Class Diagram for class Analyser]{Class Diagram for class Analyser}
    \label{fig:figure4_3}
\end{figure}

Its class diagram is in Figure 4.3, which additionally shows a parentage link to $SubAnalyser$ class for Phase IV $Sub-Analyse$. It is the only instance of inheritence in the program: $Analyser$ is used in the first iteration and its framework is re-used for code efficiency in $Sub-Analyser$ for second iteration onwards. For this subsection, only $Analyser$ is discussed; $SubAnalyser$ is to be explained in the next subsection.

Class $Analyser$ contains seven features/methods:

\begin{itemize}
	\item{maxMin()} \\
	This method is responsible for collecting (min, max) ranges for each attribute. For each of the Class value in the map, it goes through each attribute value of every corresponding tuples. A \texttt{Collections.sort()} method is applied to find the minimum and maximum value of each attribute. In the first iteration, only one range is found for each attribute.
	
	For each range, it then appends {attribute number, range} to \texttt{attributeRangeMap}. It also retrieves tuples that are covered under this range and appends {range, range-covered tuple} to \texttt{attributeTupleMap}. 
	
	These two maps, accompanied by the Class value, are passed onto the next method, \texttt{convertToBinary} to prepare binary array for max-sum analysis.
	
	\item{convertToBinary()} \\
	For each attribute in \texttt{attributeRangeMap}, this method retrieves the respective (min, max) range. Each range is used as key to search in \texttt{attributeTupleMap} to obtain tuples that are covered by this range. It then goes through each tuple to retrieve its Class tag and if this tag matches the Class value received from previous method, it appends $1$ to a new \texttt{binaryList}, otherwise appends $-1$. 
	
	This binary list is sent to the next method \texttt{maxSum()} for sub-range analysis.
	
	\item{maxSum()} \\
	For each of the binary list, this method performs a similar analysis to Kadane's max-sum algorithm: Starting with a \texttt{sum} of $0$, it traverses through the list values while adding this value to the initial \texttt{sum}, also keeping track of the element position in an \texttt{attributePosition} array. If \texttt{sum} is positive, it moves on while taking the same action. However, if it gets negative, it resets the \texttt{sum} to $0$ and also restart the \texttt{attributePosition} array. This essentially means that the current array contains too many wrong class tag if continued (i.e. too many $-1$s), and thus not necessary to proceed into more negatives. 
	
	However, while a new \texttt{attributePosition} array is created starting from this position, the current array is recorded instead of discarded, which is also considered as a sub-range ready for next iteration. This step differentiates this method from the original Kadane's since Kadane's seeks for only one optimal range, constrasting to this method where multiple sub-ranges are desired.
	
	The process is repeated until the end of array is reached. The outcome can be 1) no \texttt{attributePosition} array or sub-ranges found, 2) one \texttt{attributePosition} array or sub-range found, or 3) multiple \texttt{attributePosition} array or sub-ranges found. Regardless, if available, sub-ranges (in \texttt{attributePosition} array format) are recorded into an \texttt{List} and passed onto next method, \texttt{checkSupportConfidence()}, for threshold test. 
	
	It must be noted that, initially, whenever a \texttt{attributePosition} array is found, \texttt{checkSupportConfidence()} is called straight away so a threshold test is performed whenever \texttt{sum} turns negative. However, it means that \texttt{checkSupportConfidence()} might be called excessively in the case of a negative streak for \texttt{sum}, thus might be safer to call it after max-sum is finished. 
	
	\item{checkSupportConfidence()} \\
	 For each position array in \texttt{attributePosition}, support and confidence is measured:
	  
	 Support $\sigma$ = array size / binary list size \\
	 Confidence $\delta$ = number of positive occurence within array / array size
	 
	 Array size is calculated by the subtraction of last and first position e.g for position array $[10,21]$, the array size is $21 - 10 = 11$. Positive occurence is counted by traversing through binary values that are covered under this position range e.g for position array $[10,21]$, the program retrieves value of element number $[10]$, $[11]$,...,$[21]$ in binary list and counts the quantity of $1$s against $-1$. 
	 
	 These measures are compared to their corresponding user-specified thresholds in a conditional clause. If an array passes both thresholds, it is accepted as a candidate sub-range for next iteration. Accepted arrays are added into a new \texttt{list} to be passed onto the next method, \texttt{convertPositionToRange} where it is converted from integer position to its respective float range values.
	 
	 (Initially, no new \texttt{List} was needed as a loop was constructed so that rejected arrays were removed from the list instead of adding the accepted ones into new list. However, Java ejected \texttt{ConcurrentModificationException} as it prohibited modification when a list is in iteration, hence the change.)
	
	\item{convertPositionToRange()} \\
	For each accepted position array in the new  texttt{attributePosition}, this method converts position (integer) to its relevant range value (float). It uses \texttt{attribtuteTupleMap} to retrieve the tuples covered under this range, then seeks for the two tuples that lies at specified position. For instance, a position array $[10,21]$ represents the attribute value of tuples at position $[10]$ and $[21]$ in the relevant tuple set. Within these two tuples, attribute value is reclaimed i.e. tuple $t_1 = {0, 1.2, 1.5, 2.3, class1}$ returns attribute $1$ value as $1.2$. Thus $[10,21]$ is converted to range e.g $[1.2, 2.4]$ with this method.
	
	Once done, the converted ranges are recorded into an \texttt{attributeRangeList}. If it is in the first iteration, this list is passed onto the \texttt{appendToMap()} method in order to modify the \texttt{attributeRangeMap} and \texttt{attributeTupleMap} for the next iteration. However, if it is second iteration or above, nothing is done further although these ranges can be retrieved by a \texttt{get()} method to be sent to the next iteration. This is because for the first iteration, all attributes are processed at once and map is updated completely after loop. However, for the second iteration onwards, only one (combined) attribute is considered for each loop, thus the map is not completely changed so no need for modification. 
	
	\item{appendToMap()} \\
	This method, while only applies to the first iteration, is responsible for updating the \texttt{attributeTupleMap} and \texttt{attributeRangeMap} for the next iteration by 1) new sub-range and 2) new sub-range covered tuple set. In details, for each (converted) sub-range in \texttt{attributeRangeList}, it obtains a new sub-set of tuples that are covered by this sub-range. This sub-set of tuples are cut out from \texttt{attributeTupleMap} by using a loop that searches for tuples with attributes falling under this range. The new covered tuple set is appended to \texttt{attributeTupleMap}. The sub-range itself is also appended to \texttt{attributeRangeMap}. 
	
	\item{getAttributeRangeList()} \\
	This method is a \texttt{get()} method to grab \texttt{attributeRangeList} directly to the next iteration (only used for second iteration onwards where \texttt{appendToMap()} is skipped). 
	
\end{itemize}

\item[Sequence diagram: ] 
Below is a Sequence diagram to present data flow from $Scanners \rightarrow Analyser \rightarrow Generator$:

\begin{figure}[h]
    \centering
    \includegraphics[width=4.5in, height=5in]{figures/sequence_analyser}
    \caption[Sequence Diagram for class Analyser]{Sequence Diagram for class Analyser}
    \label{fig:figure4_32}
\end{figure}

\end{description}

\subsection{Phase III Generate}

\begin{description}

\item[Aim: ] As another core module of the program, this class concerns with the following aims:
\begin{itemize}
	\item{To combined sub-ranges found in the previous module and form larger associated ranges, by a method resembles A-priori.} 
	\item{To check if one sub-range is extendable with another by checking its attribute value.} 
	\item{To check if a candidate combination of two sub-ranges can satisfies support-confidence and extra density threshold since multiple ranges are involved.} 
	\item{To finalise the rule if no possible combination can be made i.e use the last associated ranges as final rule and send them to the last phase $Monitor$.} 
\end{itemize}

\item[Input: ] This class input is the updated range and tuple map from the previous class and the original tuple set as reference when applicable:
\begin{itemize}
	\item{\texttt{allTuple}:  all original instances in dataset.} 
	\item{\texttt{attributeRangeMap}: with attribute number as kes and new (one or more) sub-ranges as value.}
	\item{\texttt{attributeTupleMap}: with each sub-range as key and tuple set covered under sub-range as value.}
\end{itemize}

\textit{Reason: } \texttt{allTuple} is needed to be passed onto the next $SubAnalyser$ class as reference to the original dataset when applicable. \texttt{attributeRangeMap} and \texttt{attributeTupleMap} are needed as new data point for pairing associated ranges.

\item[Output: ] Firstly, $Generator$ generates new extended associated ranges and passes them as output to the next $SubAnalyser$ class in order to do range analysis again (similar to $Analyser$ content). Secondly, as long as there is possible pairing between associated ranges, $Generator$ operates a recursive call to itself that keeps generating larger associated ranges. However, if no combination is possible, $Generator$ ends the recursive call and move onto finalising the rule by calling its \texttt{finaliseRule()} method. 
  
\textit{Reason: } As $SubAnalyser$ is essentially $Analyser$ with additional density checking method, $Generator$ can repeat an iteration where 1) new sub-ranges are merged to form candidate associated ranges and 2) candidate ranges are then checked for threshold measures and if they pass the test, sub-ranges are sent to $Generator$ for another recursive call to repeat the whole iteration until no combination is possible.

\item[Data type: ] Uniformly, the new \texttt{attributeRangeMap} and \texttt{attributeTupleMap} remains to be \texttt{LinkedHashMap} with similar structure to that in $Analyser$ so that it could be recognised by Java as a compatible parameter in the next iteration.

Moreover, the \texttt{LinkedList} for attribute number and float range helps improve the efficiency in extending the new element, by simply adding new element to the end of array which is the strongest aspect of \texttt{LinkedList}.

\item[Class diagram: ] 

\begin{figure}[h]
    \centering
    \includegraphics[width=5in]{figures/class_generator}
    \caption[Class Diagram for class Generator]{Class Diagram for class Generator}
    \label{fig:figure4_4}
\end{figure}

Its class diagram is in Figure 4.4, which presents five features/methods:

\begin{itemize}
	\item{aPriori()} \\
	This method combines sub-ranges by going through each range in \texttt{attributeRangeMap} and attempts to combine it with another among the remaining ranges. When a combination is planned, the next step is to get a new attribute number combination (combinedKey) and its respective sub-range combination (combinedRange). 
	
	\item{setACombinedKey()} \\
	This method sets up the combination of attribute number e.g if attribute $1$ and $2$ has two ranges that are in pairing, a new combined key is generated to be $[1, 2]$.
	
	\item{setACombinedRange()} \\
	This method sets up the combination of range e.g if attribute $1$ has range $[1.2, 2.4]$ that is in pairing with range $[3.4,3.7]$ from attribute $2$, then a new combined ranage is generated to be ${[1.2, 2.4], [3.4, 3.7]}$.
	
	Initially, this method is merged with \texttt{setACombinedKey} method; however, it is separated for readability.
	
	\item{analyseNewCombination()} \\
	 Once new combined key and range are generated, this method sends both to the next $SubAnalyser$ class to observe whether the new (candidate) combined range could satisfies specified conditions for the next iteration. If it does, the range is returned by a \texttt{get()} method within $SubAnalyser$ to this method. 
	 
	 If the returned range is $Null$, it means threshold was not satisfied thus the program returns to \texttt{aPriori()} method to seek for new combination. However, if there is (one or more) returned range from $SubAnalyse$, this method will take the combined range and key to append it to map (similar method to \texttt{appendToMap}). A \texttt{newAttributeRangeMap} and \texttt{newAttributeTupleMap} will replace \texttt{attributeRangeMap} and \texttt{attributeTupleMap} in the last iteration respectively.
	
	\item{finaliseRule()} \\
	This method occurs only when there is no possible pairing left. A conditional clause in the $Generator$ constructor calls this method when there is only one attribute number left in \texttt{newAttributeRangeMap}. Essentially, it means all keys are combined into one largest key combination and thus no other combination is possible. It acts as a stopping point to recursive $Generator$ calls and procees to send the final combination as classification rules to the next class $Monitor$ for evaluation and display.
	
\end{itemize}

\item[Sequence diagram: ] 

Figure 4.5 shows a Sequence diagram for $Generator$ to present data flow from $Analyser \rightarrow Generator \rightarrow Monitor$:

\begin{figure}[h]
    \centering
    \includegraphics[width=4.5in, height=4in]{figures/sequence_generator}
    \caption[Sequence Diagram for class Generator]{Sequence Diagram for class Generator}
    \label{fig:figure4_42}
\end{figure}

\end{description}

\subsection{Phase IV Sub-Analyse}

\begin{description}

\item[Aim: ] As a child-class of $Analyser$, this class analyses the combined range by calling some $Analyser$ range-analysing method. It also provides extra density check and strict-adjust subset of tuples covered by candidate ranges. In other words, its aims are:
\begin{itemize}
	\item{To get tuples covered by both ranges and set it as new subset of tuples used to perform threshold tests.}
	\item{To check for density within combined ranges.} 
	\item{To recall range analysing method in parent class $Analyser$ including \texttt{convertToBinary}, \texttt{maxSum()}, \texttt{checkSupportConfidence()} and \texttt{convertPositionToRange}.} 
	\item{To return new combined range as a confirmation of combined range passing threshold check to $Generator$ through a \texttt{get()} method.}
\end{itemize}

\item[Input: ] This class input is the updated range and tuple map from the previous class, with the new combined key and combined range to analyse. Due to being a child class to $Analyser$, it has to intialises with \texttt{allClassMap} and \texttt{allTuple}. However, since they are not needed in this class, it was decided to drop them i.e. passing $Null$ as paramaters to those two in constructor:
\begin{itemize}
	\item{\texttt{allTuple}:  Now presented as $Null$.} 
	\item{\texttt{allClassMap}: Now presented as $Null$.}
	\item{\texttt{attributeTupleMap}: Same as that in $Generator$.}
	\item{\texttt{attributeRangeMap}: Same as that in $Generator$.}
	\item{\texttt{aCombinedKey}: combined key from $Generator$ to retrieve ranges when applicable.}
	\item{\texttt{aCombinedRange}: combined range from $Generator$ to apply threshold tests upon.}
\end{itemize}

\textit{Reason: } Unlike $Analyser$ in the first iteration where all attributes are considered, for the second iteration onwards with $SubAnalyser$ only one (combined) attribute and one (combined) range are considered at one time. Therefore only one (combined) range \texttt{aCombinedRange} is analysed within this class.

\item[Output: ] A new combined range is returned by a \texttt{get()} method to $Generator$ once this class is done running, if the combined range passes specified threshold tests. However, it returns $Null$ if threshold were not satisfied.
  
\textit{Reason: } As $SubAnalyser$ calls $Analyser$ method, new sub-ranges can be found from the combined range through \texttt{maxSum()} analysis. Therefore, one or multiple ranges could be returned from $Analyser$ once the methods finished. This set of range is then grabbed by $SubAnalyser$ and returned to $Generator$, indicating that there is a possible pair tested to be applicable for the next iteration. By $Analyser$ returning $Null$ when threshold requirements failed to be met, $SubAnalyser$ will also return $Null$ to $Generator$ which signifies $Generator$ to seek for new pairing.

\item[Data type: ] As discussed, using nested \texttt{LinkedList}, attribute number and range can be extended unrestrictedly. For the updated mutual tuple set, a \texttt{LinkedHashSet}. \texttt{LinkedHashSet} is used as it avoids duplication while keeping elements in order. 

\item[Class diagram: ] 

\begin{figure}[h]
    \centering
    \includegraphics[width=4in, height=2.5in]{figures/class_subanalyser}
    \caption[Class Diagram for class SubAnalyser]{Class Diagram for class SubAnalyser}
    \label{fig:figure4_5}
\end{figure}

Its class diagram is in Figure 4.5, which presents five features/methods:

\begin{itemize}
	\item{strictAdjust()} \\
 	This method seeks to construct 1) a set that contains all tuples that is at least covered by one range and 2) a mutual set of tuples where only tuples that are covered under both ranges are appended. Both sets are then passed onto next method \texttt{checkDensity()}.
	
	\item{checkDensity()} \\
	This method uses the two tuple sets from \texttt{strictAdjust()} method to perform density check on the combined range. If the combind range satisfies density requirement, it is passed onto the same range-analysis as first iteration: \texttt{convertToBinary()}, \texttt{maxSum()}, \texttt{checkSupportConfidence()} and \texttt{convertPositionToRange()} from $Analyser$ through inheritence calls. In this case density is checked first then the support and confidence is checked afterwards to reuse the code in $Analyser$. 
	
	\item{getNewCombinedRange()} \\
	This method is a \texttt{get()} method to retrieve the new combined range for $Generator$ after this class finishes.
	
	\item{getNewCombinedKey()} \\
	This method is a \texttt{get()} method to retrieve the new combined key for $Generator$ after this class finishes.
	
	\item{getNewMutualTupleList()} \\
		This method is a \texttt{get()} method to retrieve the new mutual tuple set obtained in \texttt{strictAdjust()} for $Generator$ after this class finishes.

\end{itemize}

\item[Sequence diagram: ] 

Below is a sequence diagram used to present a data flow between $Generator \rightarrow SubAnalyser \rightarrow Analyser$:

\begin{figure}[ht]
    \centering
    \includegraphics[width=5in, height=4.5in]{figures/sequence_subanalyser}
    \caption[Sequence Diagram for class SubAnalyser]{Sequence Diagram for class SubAnalyser}
    \label{fig:figure4_52}
\end{figure}

\end{description}


\subsection{Phase V Monitor}
\begin{description}

\item[Aim: ] As the final phase, this class wraps up the program with the following aims:

\begin{itemize}
	\item{To by reformating the remaining associated ranges into readable classification rules.} 
	\item{To evaluate the accuracy and performance of the rules.} 
	\item{To output the evaluation as well as the rules into a text file for the record.} 
\end{itemize}

\item[Input: ] This class input is the final version of range and tuple map from $Generator$, as well as the original dataset to retrieve range values:
\begin{itemize}
	\item{\texttt{allTuple}:  all original instances in dataset.} 
	\item{\texttt{attributeRangeMap}: with combined attribute number as key and (one or more) associated ranges (classification rules) as value.}
	\item{\texttt{attributeTupleMap}: with each associated range as key and its covered tuple set as value.}
\end{itemize}

\textit{Reason: } \texttt{allTuple} is needed to retrieve the float ranges and calculate the rule accuracy. \texttt{attributeRangeMap} and \texttt{attributeTupleMap} are the containers of the associated ranges i.e. the classification rules.

\item[Output: ] $Monitor$ produces two outputs: the evaluation of rules (including runtime and accuracy) as \texttt{String} and the rules itself to the terminal and text \texttt{.txt} file as record. 
  
\textit{Reason: } Initially, class $Monitor$ did not exist as the information is produced directly to the Terminal when the program runs. However, for readability and back-tracking functionality, a text log is decided to be kept alongside Terminal output, hence $Monitor$.  

\item[Data type: ] All information is transformed into \texttt{String} as it is appended to the text \texttt{.txt} file. 

\item[Class diagram: ] 

\begin{figure}[h]
    \centering
    \includegraphics[width=5in]{figures/class_monitor}
    \caption[Class Diagram for class Monitor]{Class Diagram for class Monitor}
    \label{fig:figure4_6}
\end{figure}

Its class diagram is in Figure 4.6, which presents five features/methods:

\begin{itemize}
	\item{evaluateRule()} \\
	This method is responsible for re-formating the ranges and calculate the rule performance and accuracy. For re-formating, due to data structure requirements the ranges found are not in correct order e.g. $r = [a,b]$ where $a > b$, so this method swaps the ranges into correct band to get $r = [b,a]$.
	 
	For rule evaluation, initially it calculated the accuracy by applying the rule onto the original dataset and observed if its prediction matched the rules. As realised at later stage, this approach was not sufficient in determining the true accuracy of the program, and training set was considered by spliting the original dataset into half. However, due to time constraint there was no working implementation for this testing part and thus \texttt{evaluateRule} only re-formats rules instead of evaluating them for the time being.
	 
	\item{outputRule()} \\
	This method outputs the rules (that are formatted into \texttt{String}) onto a text file for logging purpose. 
	
\end{itemize}

\item[Sequence diagram: ] 

A Sequence diagram is used to present the data flow between $Generator \rightarrow Monitor$:
\begin{figure}[h]
    \centering
    \includegraphics[width=3in, height=3in]{figures/sequence_monitor}
    \caption[Sequence Diagram for class Monitor]{Sequence Diagram for class Monitor}
    \label{fig:figure4_62}
\end{figure}

\end{description}

\section{Development Process}

In this section, the software engineering aspects of the project is discussed: on planning, technical tools utilised and version control for project manangement.

\subsection{Planning}

The most common way to do project planning is to rely on Gantt Chart. Gantt chart provides an overview for project development process and monitors the progress based on specified deadlines for different tasks.

However, there are algorithmic as well as implementing changes that are considered frequently during the process. It was decided that Gantt chart would not be dynamic enough to accommodate changes, and as a replacement a Kanban board was used.

Kanban gives more focus to the tasks while still keeping track of the deadlines. Trello, an online Kandan tool, is used in this project for its cross-platform mobility (as a website) and the ability to sort and archive tasks if necessary.

A screenshot of the Trello board for this project is as below:

\begin{figure}[h]
    \centering
    \includegraphics[width=5.5in]{figures/trello}
    \caption[Trello Board as Planner]{Trello Board as Planner}
    \label{fig:figure4_7}
\end{figure}


\subsection{Technical Platforms}

For this project, there are several softwares utilised:

\begin{itemize}
	\item{IntelliJ} \cite{intellij}
	
	As the first IDE used for this project, it provides a simple and convenient integration with version control from GitHub. Its SequencePlugin \cite{sequenceplugin} provides quick Sequence diagram generator that is used in this report. 
	
	\item{Eclipse} \cite{eclipse}
	
	In addition to IntelliJ, Eclipse is used in computer labs. While having similar functions as IntelliJ, Eclipse also provides a free Class diagram generator plugin (ObjectAid \cite{objectaid}) for a quick UML generator. However, the Class diagrams used in this report is generated by Visual Paradigm instead, as Visual Paradigm provides more detailed information e.g. type of parameters. 
		
	\item{Git - GitHub} \cite{github}
	
	GitHub is used for mananging the project and the report itself, which ensures mobility of the program and detailed records of code change history. GitHub is used as an integration to IntelliJ and Eclipse.
	
	GitHub links are available for the latest updates of project \cite{githubproject} and the report in both pdf and laTex format \cite{githubreport}.
	
	\item{Visual Paradigm} \cite{visualparadigm}
	
	Visual Paradigm is used to generate Class diagram in this project, replacing Eclipse plugin.
		
	\item{LaTex}
	
	LaTex is used for constructing this report. Its class template can be found at \cite{sphdtemplate}.

\end{itemize}

\subsection{Version Control}
	Version control for this project is purely done on GitHub. Since half of the project was done remotely, a distributed logging system shall be used, thus GitHub is the optimal choice for its progress tracking as well as revertible contents.
	





