\chapter{Implementation}

\section{Structure}
Based on the project algorithm design, its Java implementation unit complies with the specified structure and thus maintain the two core phases ($Analyse$ and $Generate$). However, hard code requires extra handling of data input and output. Hence additional two phases are added: $Scan$ and $Monitor$, which read and display input respectively. Moreover, the differences in data type of different iterations require extra treatment would be too clunky to append to the $Analyse$ phase. Thus, another phase is in line: $Sub-Analyse$, which inherits $Analyse$ range-analysing functions while having additional threshold check (for density) and handle elimination round for partly-covered tuples.  

The implementation process can now be divided into five Phases instead of two from the original algorithm. The reasoning is assessed in details below:

\begin{table}[!htbp]
\caption{5-Phase Implementation and Reasoning}
\label{table:table4_6}
\centering
\begin{tabular}{lc}
	\toprule
	\textbf{Reason} & \textbf{Outcome} \\
	\midrule
	To read and pre-process data input before
analysis & Phase I Scan \\
	To comply with the original structure	& Phase II Analyse \\
	To comply with the original structure	& Phase II Generate \\
	To adapt to different iteration needs	during
data analysing process & Phase IV Sub-Analyse \\
	To monitor and record the result & Phase V Monitor \\
	\bottomrule
\end{tabular} 
\end{table}


Details will be discussed in sections below:


\subsection{Phase I Scan}

\begin{description}

\item[Aim: ] To read and separate data into sub-data by class values as specified by the algorithm. 

\item[Input: ] Assumed pre-processed csv file i.e. no null values within attributes and the order of element in an instance is {attribute 1, attribute 2,..., attribute N, class C}

\textit{Reason: } Initial plan included an extra processing unit for filtering csv file when value is Null. However, time constraint did not allow the construction of this unit. Hence the programme only consider pre-processed csv data at the time the report is written.

\item[Output: ] A list of all instances of the data - a list of tuples that are presented as lists ($AllTuples$). Moreover, a partitioned map that contains class values as keys that are connected to their relevant sub-data (list of instances) as values ($AllClassMap$)
  
\textit{Reason: } $AllTuples$ is required to provide access to all original data. $AllClassMap$ is required to get the class-based data partition since the key efficiency of this algorithm is to `divide and conquer' on data partitions guided by class tag [SRC]

\item[Data type: ] \texttt{LinkedList} (for tuples), \texttt{LinkedHashMap} (for class-based data partitioned map)

\textit{Reason: } A \texttt{List} (without specified inner data type) is used for tuples since it needs multiple types of data (e.g. float number, string class value, integer order). In this case, \texttt{LinkedList} is used instead of \texttt{ArrayList} due to its ability to add and remove first and last elements more quickly than the alternative [SRC]. It is necessary to retrieve class value at faster rate as class value is the last element in the tuple.

Similarly, a \texttt{LinkedHashMap} is used for faster retrieval rate in later parts. A \texttt{HashMap} is used for storing partitioned sub-data from class values since each partition of data needs to be recorded with its unique class value. It has String class as key and \texttt{LinkedList<LinkedList>} List of all tuples as values. 

\item[Class diagram: ] [diag]

\item[Sequence diagram: ] [diag]

\end{description}




