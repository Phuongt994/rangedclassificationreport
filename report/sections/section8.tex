\chapter{Reflection}

\textbf{Theory}
	
Upon the theoretical aspect of the project, I have felt that I take too long before getting a firm grip of the algorithm. Due to the lack of experience to data mining, I take long to pick up the basics and struggle to quickly apply my understanding upon a custom method. I should have also maintained a macro-control over the algorithm (maintaining an overall picture of the data flow) instead of having a tunnel vision to individual parts of the programme. The result is my algorithm design being structurally fragmented i.e. may work well at single phase level but get confusing once iteration between phases starts.

Subsequently, most of my time is spent fixing iterative data structure errors, which could have been used for understanding and analysing the algorithm itself and possibly make improvements for it. Instead, the project is merely an implementation of the algorithm; no significant improvements were made on the theoretical front, which is not my initial expectation to the project. 

\bigskip
\textbf{Implementation}

In terms of implementation, I have chosen Java as I enjoy the precise structure required in its data. However, for the exact same reason, it does not benefit me for having a lack of exposure to data structure and algorithm in object-oriented languages. A lot of my obstacles lie in the confusion of how abstract data type (ADT) extracts its elements and how variables are kept internally within each class after forced re-initialisation in inherited classes. My persistence to maximum code reusability means I would like to keep my code as concise as possible, but this in turn forcing me to deal with inheritence within a large amount of recursive iterations. It might not worth insisting on code efficiency early in exchange for a simplier but working solution. 

\bigskip
\textbf{Project Management}

As I realise my lack of in-depth knowledge in this area, doing a research-oriented project means that I have to spend extra time in acquiring only background knowledge. However, I have been impatient with the project and often jumped in before fully understanding what it means to do this and that. My planning is therefore not as concrete as I would like, as I keep changing my mind about little details.

Moreover, I should have better risk assessment even when the programme seems to work, only to find out it gives semantically wrong output at the end of the project. 

\bigskip
Nevertheless, I thoroughly enjoy working with a research-based problem, especially in an area I would otherwise not have exposure to. I have learnt to use various technical tools and familiarised myself with new design patterns and data structures. It is an inspiring project and despite the outcome, I would like to continue doing it for pure enjoyment. 


