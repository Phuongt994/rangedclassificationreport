\chapter{Introduction}

The ability to search and extract meaningful information from a large set of data has been the core strength of data mining and machine learning research. One approach is to recognise certain patterns within a dataset, then transform these patterns into rules so later these rules could be used to predict a classifiable outcome for any given instance within that dataset. The process of transforming such patterns into rules thus requires efficient and accurate methods in order to produce correct predictions. It is important to choose the right methods and algorithms that could derive useful and relevant rules. 

\bigskip

The goal of this project is to analyse and implement an existing classification rule mining algorithm, in which rules are derived for ranged-based data. As opposed to categorical values, continuous values pose a more challenging problem since it is more difficult to determine the effective range boundaries to produce fitting predictions. Subsequently, the effectiveness depends on how well ranges are split into right intervals \cite{srikant}. To determine the right split, the algorithm implemented in this project proposes an effective splitting method, as well as guiding its search for sub-optimal ranges using class values of the dataset. This project follows a similar  structure set by these methods in its implementation.

\bigskip

In details, the existing algorithm (referred to as \textit{original algorithm} for the rest of the report) withholds two distinctive features: firstly, optimal sub-ranges are found with an approach similar to Kadane's solution to the maximum subarray problem \cite{kadane}, which is highly efficient due to its linear time complexity. Secondly, inspired by association rule mining, sub-ranges can combine for larger associated ranges with an algorithm similar to \textit{A-priori} method in generating large frequent itemsets \cite{apriori}. These two features make for a fast and efficient way to classify range-based dataset which are implemented in this project, alongside with some modification for adaption to Java data structure.





