\chapter{Introduction}
The ability to extract meaningful information from a set of data has been a crucial problem within data mining and machine learning research. One approach is to recognise certain patterns within a dataset, then transform these patterns into rules. In the future, these rules could be used to predict an classified outcome for any given instance of that dataset. The ability to transform such patterns is thus a deciding factor in producing accurate predictions. Subsequently, its accuracy relies on the choices of methods and algorithms in learning and deriving useful and relevant rules. \\

The goal of this project is to investigate an algorithm to derive classification rules for range-based data. Ranges pose a more challenging problem to assess since there are more possible patterns to deduce than that of categorical values. Therefore, before patterns can be learnt, the data must be split into `right intervals' [src] so that the rules devised will be more fitting. The process of deducing relevant subset of data thus plays a significant part in deciding the rule accuracy, and an important step before actual classification can take place. \\

In details, this project analyse and implement a method that aims to solve the above issues i.e. to classification rules for range-based numerical dataset, developed in a previously published paper by Shao, J. and Tziatzios, A. [src]. The original method withholds two distinctive features: firstly, the data trimming process is done by an algorithm similar to Kadane's [src] solving maximum-array problem, which is highly efficient due to its linear time complexity (logN). Secondly, in order to mimic that max-sum method, chosen ranges are analysed by assigning binary tags upon their covered items depending on the items' class values. By using class values, the given ranges can be narrowed down to to be more relevant to the rules derived. These two features make for a fast and efficient way to classify range-based dataset. This method is then analysed further in order to produce an improved[?] version that is implemented as a final software using Java. \\

[What was improved?]
[Outcome? Your own work?]

