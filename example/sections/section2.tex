\chapter{Background}

The project implements an algorithm originated from [SRC]. The algorithm utilises a similar approach to association rule mining: it searches for associated ranges in each attribute, then combine these ranges to form a larger rule. To make this search more efficient, class values are used to guide the search i.e. only ranges that have relevant class tags are considered. The algorithm is explained in the following sections.

% First section: Preliminaries
\section{Preliminaries}

Assuming that we have a dataset that could be presented as a table $T(A_1, A_2, A_3,..A_m, C)$, where $A_{j,m}$ as $1 \leq j \leq m$ are range attributes and $C$ is the categorical class value. A single tuple within $T$ is denoted as $t_k = (v_1,v_2,v_3..v_m,C)$ where $v_m$ is a value under $A_m$.

A sample dataset served as an example for this section can be shown in Table 2.1 below: \\

\begin{table}[h]
\caption{A sample dataset as table}
\label{table:sec1_t1}
\centering
\begin{tabular}{lccccc}
	\toprule
	\textbf{$\mathit{T}$} & \textbf{$\mathit{A_1}$} & \textbf{$\mathit{A_2}$} & \textbf{$\mathit{A_3}$} & \textbf{$\mathit{A_4}$} & \textbf{$\mathit{C}$} \\
	\midrule
	$\mathit{t1}$ & 0.75 & 1.45 & 2.13 & 4.56 & $\mathit{c_1}$ \\
	$\mathit{t2}$ & 0.64 & 1.62 & 2.64 & 4.75 & $\mathit{c_2}$ \\
	$\mathit{t3}$ & 0.71 & 1.21 & 3.11 & 3.97 & $\mathit{c_1}$ \\
	$\mathit{t4}$ & 0.57 & 1.23 & 2.75 & 4.24 & $\mathit{c_1}$ \\
	$\mathit{t5}$ & 0.80 & 1.53 & 2.34 & 4.11 & $\mathit{c_2}$ \\
	\bottomrule
\end{tabular} 
\end{table}
 

\begin{description}

\item[Definition 1 (range)]
Assume within the domain of attribute $A$ exists two values $a$ and $b$ that represents a continuous range over $A_j$, denoted $r = [a,b]_{A_j}$. This range covers a set of values in $A$ that lies between $a$ and $b$. \\
\textit{Example 1:} From Table 2.1, a range of $[0.64, 0.75]_{A_1}$ would cover a set of values $0.64$, $0.71$ and $0.75$ in $A_1$.

\bigskip

\item[Definition 2 (cover)]
Assume $r = [a,b]_{A_j}$ to be a range over attribute $A$. This range $r$ covers a set of tuples where their values are between $a$ and $b$ where $a \leq v_j \leq b$. This set of tuples that is covered by $r$ is denoted $\tau(r)$. \\
\textit{Example 2:} From Example 1, a range of $[0.64, 0.75]_{A_1}$ would cover a set of tuples $t_2, t_3, t_1$ hence its cover is $t_2, t_3, t_1$.

\bigskip

\item[Definition 3 (associated ranges)]
Assume $r_1 = [a_1,b_1]_{A_1}$ to be a range over $A_1$ and $r_2 = [a_2,b_2]_{A_2}$ to be a range over $A_2$. Those ranges are associated ranges if $\tau(r_1) \cap \tau(r_2) \neq \varnothing$ 
\textit{Example 3:} Assume $r_1 = [0.64, 0.75]_{A_1}$ and $r_2 = [1.21, 1.45]_{A_2}$. Table 2.1 shows that $r_1$ covers $t_1, t_2, t_3$ and $r_2$ covers $t_1, t_4, t_3$, which means $r_1 \cap r_2 = \{t_1, t_2, t_3\} \cap \{t_1, t_3, t_4\} = \{t_1, t_3\}$. Since there are mutual tuples between these ranges, $r_1$ and $r_2$ are associated ranges. 

\bigskip

\item[Definition 4 (range-based classification rule)]
Assume $c$ to be a class value from $C$ and $r_1, r_2, r_3..r_h$ be a set of  associated ranges. $r_1, r_2, r_3..r_h \rightarrow c$ is a range-based classification rule. \\
By definition, any tuple $t_k$ can form a rule. However, this will not be accurate nor useful. To determine whether a rule is qualified, there are criteria to be used, which are to be discussed in the sub-section below.

\bigskip

\end{description}

% Second section: Criteria
\section{Criteria}

There are three criteria used in this algorithm: support, confidence and density. This sub-section presents the formal definitions for those measures, while examples are to be provided in the next sub-section where the actual algorithm is reviewed.

\begin{description}
\item[Definition 5 (support)]
Assume $T$ to be a table and $r_1, r_2, r_3..r_h \rightarrow c$ to be a range-based classification rule derived from $T$. The support for $r$, provided $\mid . \mid$ being the size of a set, is: 
\[ \sigma(r) = \frac{\mid \tau(r_1) \cap \tau(r_2) \cap ... \cap \tau(r_h) \mid}{\mid T \mid} \] 

\item[Definition 6 (confidence)]
Assume similar settings from Definition 5, the confidence for $r$  in $T$ is:
\[ \delta(r) = \frac{\mid \tau(r_1) \cap \tau(r_2) \cap ... \cap \tau(r_h) \cap \tau(c) \mid}{\mid \tau(r_1) \cap \tau(r_2) \cap ... \cap \tau(r_h) \mid} \] 

\item[Definition 7 (density)]
Assume similar settings from Definition 5, the density for $r$ in $T$ is:
\[ \gamma(r) = \frac{\mid \tau(r_1) \cap \tau(r_2) \cap ... \cap \tau(r_h) \cap \tau(c) \mid}{\mid \tau(r_1) \cup \tau(r_2) \cup ... \cup \tau(r_h) \mid} \] 

\end{description}

% Section 3: Original Algorithm
\section{Original algorithm}

As mentioned, the algorithm provides a more efficient way to discover associated ranges than the traditional brute-force solution. Instead of traversing through all possible combinations of ranges among all attributes, this algorithm first divides the data into sub-data pools in accordance to its class value. Then, under each attribute, it searches for sub-ranges that could pass certain user-defined thresholds (confidence, support and density). These sub-ranges are then collected and ready to be tested for possible combinations between each other that passes certain thresholds similar to the previous step, which are then collected and ready to be tested for larger combinations. Additionally, after larger combinations are found, an extra adjustment step is done on class values of relevant tuples within the new associated ranges. Afterwards, new associated ranges are ready to be analysed again to find more possible sub-ranges and new combination of sub-ranges. This iteration continues until no possible combinations can be found, so that the last standing associated ranges are translated into the concluding classification rule for that class value. \\

To get a clearer structure of the algorithm, its proceedings can be divided into two phases:  \\

\begin{description}
\item[Phase I (Analyse)] 
This phase has its primary concern as to analyse ranges. Based on the algorithm description, it includes:
	\begin{enumerate}
	\item Partition data into sub-data pool using class values.
	\item For each attribute in sub-data pool, find a maximum and minimum range within that attribute and use it as a starting range for next step.
	\item From min-max starting ranges, search for sub-ranges that passes certain user-specified thresholds such as confidence, support and density. Sub-ranges are collected and passed onto Phase II to generate larger associated ranges. \\
	\end{enumerate}

\item[Phase II (Generate)] 
This phase has its primary concern as to generate larger associated ranges for the next iteration. Based on the algorithm, it includes:
	\begin{enumerate}
	\item For each sub-range obtained from last phase, check for possible combination of that range with the remaining ones.
	\item Adjust class values of relevant tuples$r_1, r_2, r_3..r_h \rightarrow c$ under the new combined sub-range.
	\end{enumerate}
\end{description}

Details for each phase are as follows:

\subsection{Phase I (Analyse)}

Given a table $T(A_1, A_2, A_3,..A_m, C)$, the algorithm attemps to seek for classification $r_1, r_2, r_3..r_h \rightarrow c_j$ by finding, among instances that are under class $c_j \subset C$, a number of associated ranges $r_1, r_2, r_3..r_h$ that have at least a minimum support ($\sigma_{min}$), density ($\theta_{min}$) and confidence ($\gamma_{min}$) that had previously been specified by the user. This process can be explained in Phase I (Analyse) in which the algorithm seeks for associated ranges and checks if thresholds are satisfied. In pseudo-code, this phase is outlined as below:











